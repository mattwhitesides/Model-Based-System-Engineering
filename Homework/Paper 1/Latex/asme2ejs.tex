\documentclass[11pt]{asme2ej}

\usepackage{epsfig}

%% The class has several options
%  onecolumn/twocolumn - format for one or two columns per page
%  10pt/11pt/12pt - use 10, 11, or 12 point font
%  oneside/twoside - format for oneside/twosided printing
%  final/draft - format for final/draft copy
%  cleanfoot - take out copyright info in footer leave page number
%  cleanhead - take out the conference banner on the title page
%  titlepage/notitlepage - put in titlepage or leave out titlepage
%  
%% The default is oneside, onecolumn, 10pt, final


\title{Making Smart Cities Smarter – MBSE Driven IoT Technical Review}

%%% first author
\author{Matthew Whitesides
    \affiliation{
	Missouri University of Science and Technology\\
    mbwxd4@umsystem.edu\\
    \today
    }	
}

\begin{document}

\maketitle

%%%%%%%%%%%%%%%%%%%%%%%%%%%%%%%%%%%%%%%%%%%%%%%%%%%%%%%%%%%%%%%%%%%%%%
\begin{abstract}
{\it 
The Internet of Things (IoT) is a perfect template for systems engineering applications. It in and of itself is a system of interconnected systems, so employing engineering techniques on the individuals and the network as a whole is a must. In the following paper, I’ll review the paper “Making Smart Cities Smarter – MBSE Driven IoT” written by Matthew Hause and James Hummell. Their technical paper goes deep into the topic of applying model-based systems engineering techniques into real-world scenarios and the challenges that someone designing a large scale IoT system may face.
} 
\end{abstract}

%%%%%%%%%%%%%%%%%%%%%%%%%%%%%%%%%%%%%%%%%%%%%%%%%%%%%%%%%%%%%%%%%%%%%%
\begin{nomenclature}
\entry{IoT}{Internet of Things.}
\entry{INCOSE}{International Council on Systems Engineering.}
\entry{MBSE}{Model Based Systems Engineering.}
\end{nomenclature}

%%%%%%%%%%%%%%%%%%%%%%%%%%%%%%%%%%%%%%%%%%%%%%%%%%%%%%%%%%%%%%%%%%%%%%
\section{Introduction}

The internet of things is a system of connected computing devices. These internet-connected devices have exploded recently in popularity in the consumer market, notably things like Amazon Echo, Google Home are now commonplace.
Also, devices previously not interconnected have added connectivity such as smart home applications, appliances, and automotive applications. 
These things are the obvious candidates most people would associate with IoT today, however, there is a large commercial, military, and civil engineering application revolution that many may have not noticed is taking place.
Countless applications can help improve productivity, reduce costs and increase intelligence for professional tasks. Some of these include interconnected manufacturing equipment, military surveillance, robotics, biometrics, etc.
There's even a DARPA let project called the "Ocean of Things" that attempts to monitor large oceanic areas through 50,000 interconnected floating sensors and it's not crazy to think one day soon there won't be a device or are not connected in some way.\\

In the paper "Making Smart Cities Smarter – MBSE Driven IoT" [1], Matthew Hause and James Hunt explore a specific type of IoT application based upon establishing a connected city infrastructure. 
Specificity they take a look at the systems engineering task of building such a large scale project using model-based systems engineering to create and manage the many different systems that a city requires and how they would interconnect.
The main goals of a "Smart City" are to reduce costs and resource consumption and to engage more effectively and actively with its citizens. This involves some key areas of improvement including, transportation, energy, health, and water and waste management.
All of these things in a large city require massive systems and complexity on their own and attempting to interconnect and improve upon them would be nearly impossible in a non-model based approach. 

%%%%%%%%%%%%%%%%%%%%%%%%%%%%%%%%%%%%%%%%%%%%%%%%%%%%%%%%%%%%%%%%%%%%%%
\section{Summary}

"Making Smart Cities Smarter – MBSE Driven IoT" delves into the process of modeling an interconnected city using model-based systems engineering techniques. 
In this paper, the authors explain how IoT systems work and how they related to potential applications in civil city engineering.
IoT is essentially systems that interconnect to other systems through a network or the internet, and each other the individual systems can benefit from model-based engineering in their way but becomes very relevant when describing how the models interact with each other.
Discussed are the various challenges involving creating a "smart" city such as security, infrastructure, and architecture, then we go into specific detail on how you would go about making models of these systems and describing how they connect.\\

The authors have a nice flow to presenting the topics at hand. 
First, they start very basic in talking about the benefits of smart products on a smaller scale, next into the infrastructure, architecture, and security of these products which gives a great introduction into how systems of IoT products work before we even get to scaling things up.
Next, they establish a good baseline of knowledge of the MBSE process and SysML and how this benefits the systems engineering process.
Finally, after we have the required knowledge they go into how this can be applied to make a smart city and in doing so give a great specific example of how the MBSE process could be applied to a traffic signal system down to the individual hardware view.\\

For the MBSE example problem, they dive into traffic control which is an intreating topic that everyone can understand but also gives a good insight into the complexity of these kinds of systems. 
In a large city, there are large instants to increase efficacy in traffic flow and minimize disruptions as much as possible, as every delay costs time and money for everyone involved and increases road ware, gas usage, and pollution. 
Introducing IoT technology can help improve the system but also can cause some concerns such as security, creating always connected versions of the city's most vital infrastructure opens up the systems to more potential security vulnerabilities.
However, using an MBSE approach we can help make sure every connection and interface in the system and sub-systems is accounted for and secure as possible.
In the end, there's a lot of interesting ideas and implications that this paper tackles, from the large idea of connecting cities and the best approach to go about engineering it.


%%%%%%%%%%%%%%%%%%%%%%%%%%%%%%%%%%%%%%%%%%%%%%%%%%%%%%%%%%%%%%%%%%%%%%
\section{Opinion}

All in all the authors present a very clear picture of what a smart city entails, starting on the basic what is IoT to how they relate to smart cities, to the challenges involved and finally detailed examples of how to model these systems of systems using SysML.
However, even though SysML is explained in detail and some history on it is told there's actually no examples of code given to show how it's used which would have been nice.\\

Also, a good job is done in explaining all aspects of the topics introduced. 
Even before we get into the specifics of MBSE, IoT and Smart Cities are all explained in concept and detail.
Within those concepts a lot of aspects and history are explained, things like SysML and security aspects of IoT are all discussed that way you have a good perspective on what to keep in mind once we get to the smart city engineering.


%%%%%%%%%%%%%%%%%%%%%%%%%%%%%%%%%
\subsection{Strengths}

It is very easy for someone who has no idea of the topic to be able to read and understand this paper which is not always the case with technical publications. 
I particularly like how at every aspect they relate the descriptions of city building to systems engineering principals and even give an example city and how it could be described in diagrams.\\

A cool thing they do is set up a real-world applicable example problem and then take a look at the MBSE approach of solving that problem from an operational view and then dig into an example systems view. 
This is very easy to follow and can easily be envisioned how it applies to a real problem you'd face in building a smart city.
Despite the kinda rough looking user diagrams once we get into the models we have a very clear picture of what's going on. 
The authors present a good first operational view that shows the different capabilities that a traffic system would have i.e. lights control, traffic monitoring, prediction, etc. 
Then within those high-level capabilities, you see properties that describe the hardware systems that make up that ability.
Next, they logically dive into the operational view when keeps the capability models but now show the interactions of what operations occur when one of those capabilities is triggered.
Like how when a traffic event happens road crews are dispatched, traffic is rerouted, etc.
Finally, we get into the systems and individual system views. 
These show how the physical hardware works to make these traffic control systems, and the interactions the hardware components have with each other.\\

On the whole, this is just about the perfect way to describe the MBSE approach to systems engineering, it's simple, clear and concise in its approach to showing how a real-life situation would work.
You can logically follow how they models work from the top view down to individual hardware, and you can easily envision this as part of a greater smart city where you are now modeling the interactions that different systems have in the city.\\

For example, refer to figure 1 above to see how they approach modeling the various views. 
You can see that each functional capability is modeled, then in the operational view it models the methods that complete the capabilities mapped out there, and finally, in the views of the system, it models the hardware systems that complete the given tasks.
It's a very logical and precise way of planning out and engineering a system like this.

\begin{figure}
    \centering
    \includegraphics[width=0.75\linewidth]{Annotation.png}
    \caption{Traffic Management Capabilities.}
    \label{fig:Traffic}
\end{figure}


%%%%%%%%%%%%%%%%%%%%%%%%%%%%%%%%%
\subsection{Potential Improvements}

This paper is more of a summary of MBSE concerning connected cities however I do feel there are a couple of areas that could be beneficial to expand upon. 
They paint a very good picture of the obvious ways that using MBSE techniques can be applied to these systems however they don't present any alternatives or suggestions on how it's currently being done. 
This would be helpful to point out how other techniques may come up short or not make as much sense while still keeping an objective mind about the proper way to plan a system.\\

Another thing I would like to see in a real-world situation where this is being applied, there are great simple examples and even some more complex diagrams however they all seem very theoretical and seeing the techniques applied in a real-world scenario would be beneficial.
Even if it would be hard to find an exact situation of a real model of an IoT city component there are likely examples of city systems that currently exist. However, their examples they do give are so clear and detailed it's hard to fault that.
It would be intreating to see how the great example systems they modeled could fit into a larger modeled view of all the city-systems interacting with each other even if just an example model.
Not that given what they have you couldn't easily picture it but it would pull the whole example together I think.

%%%%%%%%%%%%%%%%%%%%%%%%%%%%%%%%%
\subsection{Lessons Learned}

One of the key takeaways from this application of MBSE is simply the scale that can be achieved with this approach. 
If using a document-based approach to try and achieve something this large as an entire city down to the scale of how a traffic camera interacts with the light would be an impossible amount of documentation.
Not only that it'd be impossible for any person to read and be able to parse the information needed to see the connections.
However, when detailing each sub-system and component individually and connecting it to a larger system of systems you can easily see how one could insert a detailed model and instantly see the connections up through the thousands of larger models and data.\\

Something that may have been out of scope for the paper but I would have liked a bit more detail on is how we may go about defensively engineering IoT systems given a constant connection introduces a huge range of new security vulnerabilities.
One of the most obvious things someone would think about when opening up public infrastructure to connected systems and the internet is the potential for hacking which at a city-wide scale could cause massive amounts of destruction.
Also once compromised fixing these systems will not be as easy as installing a patch on a computer due to downtime and other constraints.\\

For example, a recent vulnerability found in medical devices called the URGENT/11 vulnerability [3], was a backdoor found in IoT medical devices that allowed an intruder to remotely control the device.
What's worrying about this type of problem is not that it could happen but it's not as easy to fix as it might sound, as they discovered when looking into the problem.
Like most software built today, it relies heavily on a third party and open-source software that may not be obvious how to update if a vulnerability is found, as some of the software required to update after finding this issue was maintained by maybe a single person or companies that can't quickly turn around new patches.
You can see how in a "Smart City" this would cause millions in damages if key pieces of the infrastructure were to be exposed.
Although granted using an MBSE approach you can see how that would be a more secure way of securing and detecting all the potential issues if one piece of the system was exposed.
Hopefully, you could easily see how that piece interacts and operates within the system as a whole to isolate and fix the issue.


%%%%%%%%%%%%%%%%%%%%%%%%%%%%%%%%%%%%%%%%%%%%%%%%%%%%%%%%%%%%%%%%%%%%%%
\section{Related Works}

In "Assessing the Potential of IoT in Systems Engineering Discipline" by Ramalingam and Tweten [2] they explore engineering an IoT system at scale. 
One of the big takeaways is the complexity that can quickly manifest from even a simple IoT system if you had just n interconnected devices you'd have $n(n - 1)/2$ number of connections to deal with between them.
You can easily see how this can get impossible for any one person to manage if you relate it to the idea of building a city that contains thousands of IoT devices.
Now obviously not every single IoT device interacts with every other device however managing the connections between them is a perfect application of MBSE, as you can easily model which systems interface with other systems and drill into the individual systems.\\

One of the cool things about this paper is how it takes a look at the various things that need to be considered from just a simple engineering perspective before we even start on MBSE. Everything from security, hardware, software, protocols, energy, etc. are all considered.
All those things fit into the system engineering life cycle and need to be considered at some point.
They took surveys of INCOSE professionals for what they consider the highest impact areas from a system engineering perspective with security and connectivity winning out.
Generally, this paper gives an interesting view of the unique challenges presented by an IoT system and is tenfold more important if you relate to the idea of building an IoT city were not considering these factors could have devastating impacts.

%%%%%%%%%%%%%%%%%%%%%%%%%%%%%%%%%%%%%%%%%%%%%%%%%%%%%%%%%%%%%%%%%%%%%%
\section{Conclusion}

Overall this paper introduces a lot of good concepts in a real-world situation that is easily understandable and varied. 
An IoT can easily be thought of as models of systems interacting with each other, a single device in an IoT system can have all the standard systems engineering principles applied to it, and a smart city is on a huge scale and impacts everyone.

\begin{thebibliography}{1}

    \bibitem{Hause} 
    Matthew Hause and James Hummell. 
    \textit{Making Smart Cities Smarter – MBSE Driven IoT}. 
    26th Annual INCOSE International Symposium (IS 2016) Edinburgh, Scotland, UK, 18 July 2016.

    \bibitem{Ramalingam} 
    Thirunavukkarasu Ramalingam and Daniel Tweten. 
    \textit{Assessing the Potential of IoT in Systems Engineering Discipline}. 
    Collins Aerospace, 04 December 2019.

    \bibitem{FDA} 
    United States Food and Drug Administration
    \textit{URGENT/11 Cybersecurity Vulnerabilities in a Widely-Used Third-Party Software Component May Introduce Risks During Use of Certain Medical Devices}. 
    https://www.fda.gov/medical-devices/safety-communications/urgent11-cybersecurity-vulnerabilities-widely-used-third-party-software-component-may-introduce
    FDA, 01 October 2019.
    
\end{thebibliography}

\end{document}
